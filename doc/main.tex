\documentclass{article}
\usepackage{multicol}
\usepackage{graphicx}
\title{RescueLink: A Mobile Application for Victim Localization in Emergency Situations}
\author{Paolo Palumbo, Ettore Ricci}
\begin{document}
\maketitle
\begin{abstract}
In emergency situations, rapid
and accurate localization of potential 
victims is crucial for effective rescue operations.
This project presents an innovative Android 
application using Bluetooth Low Energy (BLE) 
technology, trilateration, and GPS to enhance 
victim localization. 
The system ustilizes an ah-hoc BLE network
to share information between devices and 
computes the victim's position using trilateration 
when GPS is not available.
\end{abstract}
\begin{multicols}{2}
\section{Introduction}
During emergency situations such as natural 
disasters, fires, and large-scale accidents, 
the timely and accurate localization of 
potential victims is foundational to effective
rescue operations. 
Current methods usually rely on visual searches
and may be ineffective in low-visibility scenarios.
Due to the influence of modern technology, 
the majority of people carry smartphones and other
BLE-enabled devices with them. 
This project aims to leverage these devices to
localize victims in emergency situations.
A version of the application can also be installed
on the victim's device to further enhance the
localization process.
\section{Background}
\subsection{Bluetooth Low Energy}
Bluetooth Low Energy (BLE) is a wireless 
communication technology designed for short-range
data transmission with minimal power consumption. 
Experimental tests have shown that BLE can cover 
up to 200 meters in an open 
field\cite{Lodeiro_Santiago_2017}.
Introduced as part of the Bluetooth 4.0 
specification, BLE is optimized for applications 
requiring intermittent data transfers, this makes
it ideal for use in mobile devices.
\subsection{GPS}
Global Positioning System (GPS) is a 
satellite-based localization system that
provides accurate position information
to GPS-enabled devices.
GPS uses the travel time of signals to estimate
the distance between the device and the satellites, 
then trilateration is used to determine the 
device's position.
\subsection{Trilateration}
Trilateration is a geometric technique 
used to determine the position of a point 
by measuring the distances from that point 
to three or more known locations.
The implementation of trilateration in this
project solves an optimization problem to find
the victim's position with the lowest error.
To estimate the distance between the victim and
the BLE devices, the Received Signal Strength
(RSS) is used.
\subsection{RSS}
The Received Signal Strength (RSS) is a measure
of the power level of a received signal. 
Combined with the path loss model, and knowing the
transmit power, the RSS can be used to estimate
the distance between the transmitter and the 
receiver.
RSSI is the Received Signal Strength Indicator,
a value that represents the power level of the
received signal in dBm.
The formula used to estimate the distance is:
\begin{equation}
    D=10^{\frac{txPower - RSSI}{10 * n}}
\end{equation}
We used $n=2$ as the path loss exponent because
we expect a typical outdoor environment.
\section{System Design}
\subsection{Application Interface}
\subsection{Data Sharing}
\section{Results}
\section{Conclusion}
\end{multicols}
\bibliographystyle{IEEEtran}
\bibliography{references}
\end{document}